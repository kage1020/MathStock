\documentclass[a4paper]{ltjsarticle}

\usepackage[dvipdfmx]{graphicx}
\usepackage[dvipdfmx,hidelinks,pdfusetitle]{hyperref}
\hypersetup{
    colorlinks=false,
    bookmarksnumbered=true,
    pdfborder={0 0 0},
    bookmarkstype=toc
}
\usepackage[nobreak]{cite}
\usepackage{pxjahyper}
\usepackage{amsmath}
\usepackage{tikz}

\usetikzlibrary{datavisualization}
\usetikzlibrary{positioning}
\usetikzlibrary{shapes.geometric, shapes.misc}
\usetikzlibrary{patterns}
\usetikzlibrary{calc}

\begin{document}

% ====================

\begin{itembox}[l]{東京工業大学 1990年}
    $x$,$y$,$z$,$w$ を正数とする.任意の正の整数 $m$,$n$ に対して,

    \begin{equation*}
        \left(x^\frac{1}{m}+y^\frac{1}{m}\right)^n+\left(z^\frac{1}{m}+w^\frac{1}{m}\right)^n=\left\{\left(x^\frac{n}{m}+z^\frac{n}{m}\right)^\frac{1}{n}+\left(y^\frac{n}{m}+w^\frac{n}{m}\right)^\frac{1}{n}\right\}^n
    \end{equation*}

    が成り立つための必要十分条件を求めよ.
\end{itembox}

$x^{1/m}=X$,$y^{1/m}=Y$,$z^{1/m}=Z$,$w^{1/m}=W$ とおくと,

\begin{equation}
    (X+Y)^n+(Z+W)^n=\left\{\left(X^n+Z^n\right)^\frac{1}{n}+\left(Y^n+W^n\right)^\frac{1}{n}\right\}^n\label{eq:1}
\end{equation}

となる。\eqref{eq:1}式が任意の正の整数で成立するならば,$n=2$ のときも\eqref{eq:1}式が成立することが必要である.すなわち,

\begin{align}
    (X+Y)^2+(Z+W)^2 & =\left\{(X^2+Z^2)^\frac{1}{2}+(Y^2+W^2)^\frac{1}{2}\right\}^2\nonumber \\
    XY+ZW           & =\sqrt{X^2+Z^2}\sqrt{Y^2+W^2}\nonumber                                 \\
    (XY+ZW)^2       & =(X^2+Z^2)(Y^2+W^2)\nonumber                                           \\
    (XW-YZ)^2       & =0\nonumber                                                            \\
    \therefore XW   & =YZ\label{eq:2}
\end{align}

が必要である.

逆に\eqref{eq:2}式のとき,正の実数 $k$ を用いて $Z=kX$,$W=kY$ とおくことができ,このとき,

\begin{align*}
    (X+Y)^n+(Z+W)^n                                                                    & =(X+Y)^n+k^n(X+Y)^n=(1+k^n)(X+Y)^n                                                                                                                   \\
    \left\{\left(X^n+Z^n\right)^\frac{1}{n}+\left(Y^n+W^n\right)^\frac{1}{n}\right\}^n & =\left[\left\{(1+k^n)X^n\right\}^\frac{1}{n}+\left\{(1+k^n)Y^n\right\}^\frac{1}{n}\right]^n=\left\{(1+k^n)^\frac{1}{n}(X+Y)\right\}^n=(1+k^n)(X+Y)^n
\end{align*}

したがって\eqref{eq:1}式は任意の正の整数 $n$ に対して成立する.

以上から,求める条件は,任意の正の整数 $m$ に対して $XW=YZ$ が成り立つ.すなわち,$xw=yz$ である.

% ====================

\begin{itembox}[l]{東京工業大学 1990年後期}
    \begin{enumerate}[label=(\arabic*)]
        \item $n-1$ 次多項式 $\mathrm{P}_n(x)$ と $n$ 次多項式 $\mathrm{Q}_n(x)$ ですべての実数 $\theta$ に対して

              \begin{align*}
                  \sin(2n\theta) & =n\sin(2\theta)\mathrm{P}_n(\sin^2\theta) \\
                  \cos(2n\theta) & =\mathrm{Q}_n(\sin^2\theta)
              \end{align*}

              を満たすものが存在することを帰納法を用いて示せ.

        \item $k=1, 2, \ldots, n-1$ に対して,$\alpha_k=\left(\sin\dfrac{k\pi}{2n}\right)^{-2}$ とおくと,

              \begin{equation*}
                  \mathrm{P}_n(x)=(1-{\alpha_1}x)(1-{\alpha_2}x)\cdots(1-{\alpha_{n-1}}x)
              \end{equation*}

              となることを示せ.

        \item $\displaystyle\sum_{k=1}^{n-1}\alpha_k=\dfrac{2n^2-2}{3}$ を示せ.
    \end{enumerate}
\end{itembox}

\begin{enumerate}[label=(\arabic*)]
    \item $n-1$ 次多項式 $\mathrm{P}_n(x)$ と $n$ 次多項式 $\mathrm{Q}_n(x)$ ですべての実数 $\theta$ に対して,

          \begin{equation*}
              \sin(2n\theta)=n\sin(2\theta)\mathrm{P}_n(\sin^2\theta),\ \cos(2n\theta)=\mathrm{Q}_n(\sin^2\theta)\tag{\ast}
          \end{equation*}

          を満たすものが存在する.これを数学的帰納法で示す.
          \begin{enumerate}[label=(\Roman*)]
              \item $n=2$ のとき

                    \begin{align*}
                        \sin 4\theta & =2\sin{2\theta}\cos{2\theta}=2\sin2\theta(1-2\sin^2\theta)  \\
                        \cos 4\theta & =1-2\sin^{2}2\theta=1-2\cdot 4\sin^2\theta\cdot\cos^2\theta \\
                                     & =1-8\sin^2\theta(1-\sin^2\theta)                            \\
                                     & =8\sin^4\theta-8\sin^2\theta+1
                    \end{align*}

                    したがって,$\mathrm{P}_2(x)=1-2x$,$\mathrm{Q}_2(x)=8x^2-8x+1$ とすると,$n=2$ のとき ($\ast$) は成立する.

              \item $n=k(\geqq2)$ のとき $(\ast)$ が成り立つと仮定する.

                    \begin{align*}
                        \sin 2(k+1)\theta & =\sin(2k\theta+2\theta)                                                                                   \\
                                          & =\sin{2k\theta}\cos 2\theta+\cos{2k\theta}\sin2\theta                                                     \\
                                          & =k\sin{2\theta}\cdot\mathrm{P}_k(\sin^2\theta)\cdot(1-\sin^2\theta)+\mathrm{Q}_k(\sin^2\theta)\sin2\theta \\
                        \cos 2(k+1)\theta & =\cos(2k\theta+2\theta)                                                                                   \\
                                          & =\cos{2k\theta}\cos2\theta-\sin{2k\theta}\sin2\theta                                                      \\
                                          & =\mathrm{Q}_k(\sin^2\theta)(1-2\sin^2\theta)-k\sin 2\theta\cdot\mathrm{P}_k(\sin^2\theta)\cdot\sin2\theta \\
                                          & =\mathrm{Q}_k(\sin^2\theta)(1-2\sin^2\theta)-k(2\sin\theta\cos\theta)^2\cdot\mathrm{P}_k(\sin^2\theta)    \\
                                          & =\mathrm{Q}_k(\sin^2\theta)(1-2\sin^2\theta)-4k\sin^2\theta(1-\sin^2\theta)\mathrm{P}_k(\sin^2\theta)
                    \end{align*}

                    したがって

                    \begin{equation*}
                        \left\{\begin{aligned}
                            \mathrm{P}_{k+1}(x) & =\frac{1}{k+1}\qty{k(1-2x)\mathrm{P}_k(x)+\mathrm{Q}_k(x)} \\
                            \mathrm{Q}_{k+1}(x) & =(1-2x)\mathrm{Q}_k(x)-4kx(1-x)\mathrm{P}_k(x)
                        \end{aligned}\right.
                    \end{equation*}

                    とおくと,$\mathrm{P}_{k+1}(x)$ は $k$ 次以下,$\mathrm{Q}_{k+1}(x)$ は $k+1$ 次以下の多項式となる.

                    ここで, $\mathrm{P}_k(x)$,$\mathrm{Q}_k(x)$ の最高次の項をそれぞれ $a_{k}x^{k-1}$,$b_{k}x^k$ とおくと\eqref{eq:1}式, \eqref{eq:2}式より

                    \begin{equation}
                        \begin{dcases}
                            a_{k+1}=\dfrac{1}{k+1}(-2ka_k+b_k) \\
                            b_{k+1}=-2b_k+4ka_k
                        \end{dcases}\label{eq:3}
                    \end{equation}

                    である.このとき,$a_2=-2<0$,$b_2=8>0$ である.\eqref{eq:3}式より,

                    \begin{align*}
                         & a_k>0,\ b_k<0\text{ときは} & a_{k+1}<0,\ b_{k+1}>0 \\
                         & a_k<0,\ b_k>0\text{ときは} & a_{k+1}>0,\ b_{k+1}<0
                    \end{align*}

                    となる.したがって,$a_{k+1}\neq 0$,$b_{k+1}\neq 0$ であるから\eqref{eq:1}式, \eqref{eq:2}式より,

                    $\mathrm{P}_{k+1}(x)$ は $x$ の $k$ 次多項式,$\mathrm{Q}_{k+1}(x)$ は $x$ の $k+1$ 次多項式となり,($\ast$) は成立する。
          \end{enumerate}

          以上,(I), (II)より,2以上の全ての自然数 $n$ に対して ($\ast$) は成立する.

    \item $1/\alpha_k=\sin^2\qty(k\pi/2n)\ (k=1, 2, \cdots, n-1)$ は互いに相異なる $n-1$ 個の数である.

          $\theta_k=k\pi/2n$ とおくと,$\sin2n\theta_k=0$ であり,$0<\theta_k<\pi/2$ より,$\sin2\theta_k\neq0$ となる.

          よって(1)より

          \begin{align*}
              \sin 2n\theta_k                         & =n\sin{2\theta_k}\mathrm{P}_n(\sin^2\theta_k) \\
              \therefore 0                            & =n\sin{2\theta_k}\mathrm{P}_n(\sin^2\theta_k) \\
              \therefore \mathrm{P}_n(\sin^2\theta_k) & =0
          \end{align*}

          となる.したがって,$n-1$ 次方程式 $\mathrm{P}_n(x)$ は,

          \begin{equation*}
              \sin^2\theta_k \qty(=\dfrac{1}{\alpha_k})\ (k=1, 2, \cdots, n-1)
          \end{equation*}

          を解にもつ.よって,$A$ を定数として,

          \begin{equation*}
              \mathrm{P}_n(x)=A(1-\alpha_{1}x)(1-\alpha_{2}x)\cdots(1-\alpha_{n-1}x)
          \end{equation*}

          とおける.

          また,$\mathrm{P}_n(x), \mathrm{Q}_n(x)$ の定数項をそれぞれ $c_n, d_n$ とすると,\eqref{eq:1}式,\eqref{eq:2}式より,

          \begin{equation*}
              c_{n+1}=\dfrac{1}{n+1}(nc_n+d_n),\ d_{n+1}=d_n
          \end{equation*}

          であり,$c_2=d_2=1$ であるから,帰納的に2以上の全ての整数 $n$ に対して $c_n=d_n=1$

          したがって $A=1$ である.

          以上から,$\mathrm{P}_n(x)=(1-\alpha_{1}x)(1-\alpha_{2}x)\cdots(1-\alpha_{n-1}x)$ となる.

    \item $\mathrm{P}_n(x)=1-(\alpha_1+\alpha_2+\cdots+\alpha_{n-1})x+$(2次以上の多項式)

          よって $e_n=\displaystyle\sum_{k=1}^{n-1}\alpha_k$ とおくと,

          \begin{equation*}
              \mathrm{P}_n(x)=1-e_{n}x+(\text{2次以上の多項式})
          \end{equation*}

          また,$\mathrm{Q}_n(x)$ の定数項は1であるから,

          \begin{equation*}
              \mathrm{Q}_n(x)=1+f_{n}x+(\text{2次以上の多項式})
          \end{equation*}

          と表せる.\eqref{eq:1}式より

          \begin{align}
              (n+1)\mathrm{P}_{n+1}(x)= & n(1-2x)\mathrm{P}_{n}(x)+\mathrm{Q}_{n}(x)\nonumber \\
              =                         & n(1-2x)\qty{1-e_{n}x+(\text{2次以上の多項式})}\nonumber    \\
                                        & +1+f_{n}x+(\text{2次以上の多項式})\nonumber                \\
              =                         & n+1+(-2n-ne_n+f_n)x+(\text{2次以上の多項式})\nonumber      \\
              \therefore -(n+1)e_{n+1}= & -2n-ne_n+f_n\label{eq:4}
          \end{align}

          さらに\eqref{eq:2}式より

          \begin{align}
              \mathrm{Q}_{n+1}(x)= & (1-2x)\qty{1+f_{n}x+(\text{2次以上の多項式})}\nonumber    \\
                                   & -4nx(1-x)\qty{1-e_{n}x+(\text{2次以上の多項式})}\nonumber \\
              =                    & 1+(-2+f_n-4n)x+(\text{2次以上の多項式})\nonumber          \\
              \therefore f_{n+1}=  & -2+f_n-4n=f_n-2(2n+1)\label{eq:5}
          \end{align}

          (1)より,$f_2=-8$ であるから,$n\geqq 3$ において

          \begin{equation*}
              f_n=f_2+\displaystyle\sum_{l=2}^{n-2}\qty{-2(2l+1)}=-2n^2
          \end{equation*}

          これは $n=2$ のときも正しい.したがって\eqref{eq:3}式より,

          \begin{align*}
              -(n+1)e_{n+1}           & =-ne_n-2n^2-2n \\
              \therefore (n+1)e_{n+1} & =ne_n+2n(n+1)
          \end{align*}

          (1)より $e_2=2$ であるから,$n\geqq 3$ において

          \begin{align*}
              ne_n           & =2e_2+\displaystyle\sum_{l=2}^{n-1}2l(l+1) \\
                             & =4+\frac{2}{3}\qty{(n-1)n(n+1)-6}          \\
                             & =\frac{2}{3}(n-1)n(n+1)                    \\
              \therefore e_n & =\frac{2}{3}(n-1)(n+1)=\frac{1}{3}(2n^2-2)
          \end{align*}

          これは $n=2$ のときも正しい.

          以上から $\displaystyle\sum_{k=1}^{n-1}\alpha_k=e_n=\frac{2n^2-2}{3}$ が成り立つ.\newline

          ※\eqref{eq:1}式, \eqref{eq:2}式の辺々を $x$ で微分して,$c_n=d_n=1$ より

          \begin{equation}
              \begin{dcases}
                  \mathrm{P}_{n+1}'(0)=\frac{-2u}{n+1}+\frac{n}{n+1}\mathrm{P}_{n}'(0)+\frac{1}{n+1}\mathrm{Q}_{n}'(0) \\
                  \mathrm{Q}_{n+1}'(0)=-2+\mathrm{Q}_{n}'(0)-4n
              \end{dcases}\label{eq:6}
          \end{equation}

          $\mathrm{Q}_n'(0)=-2n^2$ より $\displaystyle\sum_{k=1}^{n-1}\alpha_k=-\mathrm{P}_n'(0)=\frac{2n^2-2}{3}$
\end{enumerate}

\end{document}
