\documentclass[a4paper]{ltjsarticle}

\usepackage[dvipdfmx]{graphicx}
\usepackage[dvipdfmx,hidelinks,pdfusetitle]{hyperref}
\hypersetup{
    colorlinks=false,
    bookmarksnumbered=true,
    pdfborder={0 0 0},
    bookmarkstype=toc
}
\usepackage[nobreak]{cite}
\usepackage{pxjahyper}
\usepackage{amsmath}
\usepackage{tikz}

\usetikzlibrary{datavisualization}
\usetikzlibrary{positioning}
\usetikzlibrary{shapes.geometric, shapes.misc}
\usetikzlibrary{patterns}
\usetikzlibrary{calc}

\begin{document}

% ====================

\begin{itembox}[l]{東京工業大学 1996年}
    2以上の整数 $n$ に対して方程式 $x_1+x_2+\cdots+x_n=x_{1}x_{2}\cdots x_{n}$ の正の整数解 $(x_1,\ x_2,\ \cdots,\ x_n)$ を考える.ただし,例えば $(1,\ 2,\ 3)$ と $(3,\ 2,\ 1)$ は異なる解とみなす.このとき,次の問いに答えよ.

    \begin{enumerate}[label=(\arabic*)]
        \item $n=2$ および $n=3$ のときの解をすべて求めよ.

        \item 解が1つしかないような $n$ をすべて答えよ.

        \item 任意の $n$ に対して解は少なくとも1つ存在し,かつ有限個しかないことを示せ.
    \end{enumerate}
\end{itembox}

\begin{enumerate}[label=(\arabic*)]
    \item $n=2$ のとき

          \begin{equation*}
              x_1+x_2=x_{1}x_{2} \quad \therefore (x_1-1)(x_2-1)=1
          \end{equation*}

          よって解は $(x_1,\ x_2)=(2,\ 2)$ である.

          $n=3$ のとき

          $x_1\leqq x_2\leqq x_3$ と仮定すると,

          \begin{equation*}
              \left\{\begin{aligned}
                  x_1+x_2+x_3     & \leqq 3x_3           \\
                  x_{1}x_{2}x_{3} & \geqq {x_{1}}^{2}x_3
              \end{aligned}\right.
          \end{equation*}

          であるから,$x_1+x_2+x_3=x_{1}x_{2}x_{3}$ となるには,

          \begin{equation*}
              3x_3\geqq {x_{1}}^{2}x_3 \quad \therefore{x_{1}}^{2}\leqq 3
          \end{equation*}

          すなわち,$x_1=1$ であることが必要となる.このとき,

          \begin{equation*}
              1+x_2+x_3=x_2x_3 \quad \therefore (x_2-1)(x_3-1)=2
          \end{equation*}

          よって解は $(x_2,\ x_3)=(2,\ 3)$ となる.$x_1\leqq x_2\leqq x_3$ の仮定をはずして,求める解は,

          \begin{equation*}
              (x_1,\ x_2,\ x_3)=(1,\ 2,\ 3),\ (1,\ 3,\ 2),\ (2,\ 1,\ 3),\ (2,\ 3,\ 1),\ (3,\ 1,\ 2),\ (3,\ 2,\ 1)
          \end{equation*}

    \item 解が1つしか存在しないとき,その解は,

          \begin{equation*}
              (x_1,\ x_2,\ \cdots,\ x_n)=(k,\ k,\ \cdots,\ k)\quad (k\text{は正の整数})
          \end{equation*}

          の形である.このとき,$nk=k^n$

          $k=1$ とすると $n=1$ となり,$n\geqq 2$ に反するから $k\geqq 2$ である.二項定理より,

          \begin{align*}
              k^n & =\{1+(k-1)\}^n                                                    \\
                  & =_{n}C_0+_{n}C_{1}(k-1)+_{n}C_{2}(k-1)^2+\cdots +_{n}C_{n}(k-1)^n \\
                  & \geqq _{n}C_0+_{n}C_{1}(k-1)+_{n}C_{2}(k-1)^2                     \\
                  & =1+n(k-1)+\frac{1}{\ 2\ }n(n-1)(k-1)^2
          \end{align*}

          であるから,

          \begin{align*}
              nk                & \geqq 1+n(k-1)+\frac{1}{\ 2\ }n(n-1)(k-1)^2 \\
              \therefore 2(n-1) & \geqq n(n-1)(k-1)^2                         \\
              \therefore 2      & \geqq n(k-1)^2
          \end{align*}

          $k,\ n$ はともに2以上であるから,$k=n=2$ となる.

          逆に,$n=2$ のとき,(1)より解はただ1組となる.よって,求める $n$ は,$n=2$ である.

    \item $\lbrack$ 任意の $n$ に対して解は少なくとも1つ存在すること $\rbrack$

          $x_1=x_2=\cdots =x_{n-2}=1, x_{n-1}=2, x_n=n$ とすると,

          \begin{equation*}
              \left\{\begin{aligned}
                  x_1+x_2+\cdots+x_{n-2}+x_{n-1}+x_n=(n-2)+2+n & =2n \\
                  x_{1}x_{2}\cdots x_{n-2}x_{n-1}x_n           & =2n
              \end{aligned}\right.
          \end{equation*}

          となるから,解は少なくとも1つ存在する.

          $\lbrack$ 解は有限個しかないこと $\rbrack$

          $x_1\leqq x_2\leqq\cdots\leqq x_n$ という大小関係を仮定する.

          \begin{equation*}
              \left\{\begin{aligned}
                  x_1+x_2+\cdots+x_{n-1}+x_n    & \leqq nx_n         \\
                  x_{1}x_{2}\cdots x_{n-1}x_{n} & \geqq x_{n-1}x_{n}
              \end{aligned}\right.
          \end{equation*}

          であり,解が存在するには,

          \begin{equation*}
              \quad x_{n-1}x_n\leqq nx_n \qquad \therefore x_{n-1}\leqq n
          \end{equation*}

          が必要である.よって $(x_1,\ x_2,\ \cdots,\ x_{n-1})$ の組は $n^{n-1}$ 個しかない.

          一方,

          \begin{equation}
              x_1+x_2+\cdots+x_n-1+x_n=x_{1}x_{2}\cdots x_{n-1}x_{n}\label{eq:1}
          \end{equation}

          は $x_1=x_2=\cdots=x_{n-1}=1$ とすると,$(n-1)+x_n=x_n$ となり,$n\geqq 2$ よりこれを満たす $x_n$ は存在しない.

          よって,

          \begin{equation*}
              2\leqq x_{1}x_{2}\cdots x_{n-1}
          \end{equation*}

          したがって\eqref{eq:1}式は,

          \begin{align*}
              (x_{1}x_{2}\cdots x_{n-1}-1)x_{n} & =x_1+x_2+\cdots+x_{n-1}                                        \\
              x_n                               & =\frac{\ x_1+x_2+\cdots+x_{n-1}\ }{x_{1}x_{2}\cdots x_{n-1}-1}
          \end{align*}

          となり,$x_1,\ x_2,\ \cdots,\ x_{n-1}$ の値が決定すると $x_n$ は1つに決まる.

          したがって,$x_1\leqq x_2\leqq \cdots\leqq x_n$ のとき解は有限であり,この条件を外しても解の個数は $n!$ 倍になるだけで,有限個である.
\end{enumerate}

% ====================

\end{document}
