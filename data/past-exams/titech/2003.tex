\documentclass[a4paper]{ltjsarticle}

\usepackage[dvipdfmx]{graphicx}
\usepackage[dvipdfmx,hidelinks,pdfusetitle]{hyperref}
\hypersetup{
    colorlinks=false,
    bookmarksnumbered=true,
    pdfborder={0 0 0},
    bookmarkstype=toc
}
\usepackage[nobreak]{cite}
\usepackage{pxjahyper}
\usepackage{amsmath}
\usepackage{tikz}

\usetikzlibrary{datavisualization}
\usetikzlibrary{positioning}
\usetikzlibrary{shapes.geometric, shapes.misc}
\usetikzlibrary{patterns}
\usetikzlibrary{calc}

\begin{document}

% ====================

\begin{itembox}[l]{東京工業大学 2003年 後期}
    $m$ を0以上の整数とする.直線 $2x+3y=m$ 上の点 $(x,\ y)$ で,$x$,$y$ がともに0以上の整数であるものの個数を $N(m)$ とする.

    \begin{enumerate}[label=(\arabic*)]
        \item $N(m+6)=N(m)+1$ を証明せよ.
        \item $N(m)=1-m+\qty[\dfrac{m}{2}]+\qty[\dfrac{2m}{3}]$ を証明せよ.ただし,$[a]$ は $a$ 以下の最大の整数を表すものとする.
    \end{enumerate}
\end{itembox}

\begin{enumerate}[label=(\arabic*)]
    \item
          \begin{equation}
              2x+3y=m\label{eq:1}
          \end{equation}

          $(x,\ y)=(-m,\ m)$ は\eqref{eq:1}式を満たすから,

          \begin{equation}
              2(-m)+3m=m\label{eq:2}
          \end{equation}

          \eqref{eq:1}式, \eqref{eq:2}式の辺々を引いて,

          \begin{equation*}
              2(x+m)+3(y-m)=0 \quad \therefore 2(x+m)=-3(y-m)
          \end{equation*}

          2と3は互いに素であるから,$t$ を整数として

          \begin{equation*}
              x+m=3t, y-m=-2t \quad \therefore x=-m+3t, y=m-2t
          \end{equation*}

          $x\geqq 0$,$y\geqq 0$ より $-m+3t\geqq 0$,$m-2t\geqq 0$

          \begin{equation}
              \therefore \frac{m}{3}\leqq t\leqq \frac{m}{2}\label{eq:3}
          \end{equation}

          \eqref{eq:3}式を満たす整数 $t$ の個数が $N(m)$ である.\eqref{eq:3}式の $m$ を $m+6$ として,

          \begin{equation*}
              \frac{m}{3}+2\leqq t\leqq \frac{m}{2}+3
          \end{equation*}

          $m/3+2\leqq t\leqq m/2+3$ を満たす $t$ の個数は\eqref{eq:3}式を満たす $t$ の個数,すなわち,$N(m)$ に等しく,$m/2+2<t\leqq m/2+3$ を満たす $t$ はちょうど1個である.したがって,

          \begin{equation*}
              N(m+6)=N(m)+1
          \end{equation*}

          が成り立つ.

    \item \eqref{eq:3}式より,$N(0)=1$,$N(1)=0$,$N(2)=1$,$N(3)=1$,$N(4)=1$,$N(5)=1$

          ここで,$X(m)=1-m+\qty[m/2]+\qty[2m/3]$ とおくと,

          \begin{equation*}
              X(0)=1,\ X(1)=0,\ X(2)=1,\ X(3)=1,\ X(4)=1,\ x(5)=1
          \end{equation*}

          よって,$m=0,\ 1,\ 2,\ 3,\ 4,\ 5$ のとき $N(m)=X(m)$ が成り立つ.

          また,ある $m$ に対して,$N(m)=X(m)$ が成り立つとすると,

          \begin{align*}
              X(m+6) & =1-(m+6)+\qty[\frac{m}{2}+3]+\qty[\frac{2m}{3}+4] \\
                     & =1-(m+6)+\qty[\frac{m}{2}]+3+\qty[\frac{3m}{2}]+4 \\
                     & =2-m+\qty[\frac{m}{2}]+\qty[\frac{2m}{3}]         \\
                     & =X(m)+1
          \end{align*}

          (1)の結果と合わせて $N(m+6)=X(m+6)$ が成り立つ.

          以上から,0以上の全ての $m$ に対して $N(m)=X(m)$ が成り立つ.
\end{enumerate}

% ====================

\end{document}
