\documentclass[a4paper]{ltjsarticle}

\usepackage[dvipdfmx]{graphicx}
\usepackage[dvipdfmx,hidelinks,pdfusetitle]{hyperref}
\hypersetup{
    colorlinks=false,
    bookmarksnumbered=true,
    pdfborder={0 0 0},
    bookmarkstype=toc
}
\usepackage[nobreak]{cite}
\usepackage{pxjahyper}
\usepackage{amsmath}
\usepackage{tikz}

\usetikzlibrary{datavisualization}
\usetikzlibrary{positioning}
\usetikzlibrary{shapes.geometric, shapes.misc}
\usetikzlibrary{patterns}
\usetikzlibrary{calc}

\begin{document}

% ====================

\begin{itembox}[l]{東京工業大学 1984年}
    自然数 $a,\ b,\ c$ が $3a=b^3,\ 5a=c^2$ を満たし,$d^6$ が $a$ を割り切るような自然数 $d$ は $d=1$ に限るとする.

    \begin{enumerate}[label=(\arabic*)]
        \item $a$ は3と5で割り切れることを示せ.

        \item $a$ の素因数は3と5以外にないことを示せ.

        \item $a$ を求めよ.
    \end{enumerate}
\end{itembox}

\begin{enumerate}[label=(\arabic*)]
    \item
          \begin{align}
              3a & =b^3\label{eq:1} \\
              5a & =c^2\label{eq:2}
          \end{align}

          \eqref{eq:1}式より,$b^3$ は3で割り切れる.3は素数であるから,$b$ は3で割り切れる.よって,\eqref{eq:1}式の右辺は $3^3$ を因数にもつから,$a$ は $3^3$ で割り切れる.

          同様に,\eqref{eq:2}式より,$c$ は5で割り切れる.よって,\eqref{eq:2}式の右辺は $5^2$ を因数にもつから,$a$ は5で割り切れる.

          以上から,$a$ は3と5で割り切れる.

    \item $a$ が3と5以外の素因数 $p$ をもつとし,$a$ を素因数分解したときの $p$ の指数を $m$ とする.

          \eqref{eq:1}式より,$b$ は $p$ を素因数にもつから,$b$ を素因数分解したときの$p$ の指数を $l$ とすると,$m=3l$ となる.また,\eqref{eq:2}式より,$c$ も $p$ を素因数にもつから,$c$ を素因数分解したときの $p$ の指数を $k$ とすると,$m=2k$ となる.

          よって,$m$ は2,3で割り切れるから,$m$ は6の倍数となる.

          したがって,$a$ は $p^6$ で割り切れるが,これは $d^6$ が $a$ を割り切るような自然数 $d$ は $d=1$ に限ることに反する.

          以上から,$a$ の素因数は3と5以外にない.

    \item (1),\ (2)から,

          \begin{equation*}
              a=3^{x}5^{y}\quad (x,\ y\text{は5以下の自然数})
          \end{equation*}

          と表され,\eqref{eq:1}式より $x+1,\ y$ はともに3で割り切れ,\eqref{eq:2}式より $x,\ y+1$ はともに2で割り切れるから,

          \begin{equation*}
              x=2, y=3
          \end{equation*}

          となる.よって,$a=3^2\cdot 5^3=1125$
\end{enumerate}

% ====================

\end{document}
