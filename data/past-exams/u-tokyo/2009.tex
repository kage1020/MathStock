\documentclass[a4paper]{ltjsarticle}

\usepackage[dvipdfmx]{graphicx}
\usepackage[dvipdfmx,hidelinks,pdfusetitle]{hyperref}
\hypersetup{
    colorlinks=false,
    bookmarksnumbered=true,
    pdfborder={0 0 0},
    bookmarkstype=toc
}
\usepackage[nobreak]{cite}
\usepackage{pxjahyper}
\usepackage{amsmath}
\usepackage{tikz}

\usetikzlibrary{datavisualization}
\usetikzlibrary{positioning}
\usetikzlibrary{shapes.geometric, shapes.misc}
\usetikzlibrary{patterns}
\usetikzlibrary{calc}

\begin{document}

% ====================

\begin{itembox}[l]{東京大学 2009年 文科 第1問}
    座標平面において原点を中心とする半径2の円を $C_1$ とし,点 $(1,\ 0)$ を中心とする半径1の円を $C_2$ とする.また,点 $(a,\ b)$ を中心とする半径 $t$ の円 $C_3$ が, $C_1$ に内接し,かつ $C_2$ に外接すると仮定する.ただし, $b$ は正の実数とする.

    \begin{enumerate}[label=(\arabic*)]
        \item $a,\ b$ を $t$ を用いて表せ.また,$t$ が取り得る値の範囲を求めよ.

        \item $t$ が(1)で求めた範囲を動くとき, $b$ の最大値を求めよ.
    \end{enumerate}
\end{itembox}

\begin{enumerate}[label=(\arabic*)]
    \item $C_2$ の中心を $O_2$,$C_3$ の中心を $O_3$ とし, $C_1$ と $C_3$ の接点を $A$,$C_3$ の接点を $B$ とする.

          $O$,$O_3$ および $A$ は一直線上にあるので,

          \begin{align}
              \text{OO}_3             & =2-t\label{eq:1}     \\
              \therefore\quad a^2+b^2 & =(2-t)^2\label{eq:2}
          \end{align}

          また,$O_2$,$B$ および $O_3$ は一直線上にあるので,

          \begin{align}
              \text{O}_2\text{O}_3        & =t+1\label{eq:3}     \\
              \therefore\quad (a-1)^2+b^2 & =(t+1)^2\label{eq:4}
          \end{align}

          \eqref{eq:2}$-$\eqref{eq:4} より,

          \begin{align*}
              2a-1              & =-6t+3 \\
              \therefore\quad a & =-3t+2
          \end{align*}

          上式を\eqref{eq:2}に代入すると,

          \begin{align}
              b^2=(2-t)^2-(-3t+2)^2 & =-8t^2+8t\label{eq:5}     \\
              \therefore\quad b     & =\sqrt{-8t^2+8t}\nonumber
          \end{align}

          \eqref{eq:1},\eqref{eq:3}および\eqref{eq:5}がすべて正になる条件より,$0<t<1$

    \item $f(t)=b^2$ を考える.

          \begin{equation*}
              f(t)=-8t^2+8t=-8\left(t-\frac{1}{2}\right)^2+2
          \end{equation*}

          $0<t<1$ より,$f(t)$ は $t=\dfrac{1}{2}$ で最大値2をとる.このとき,$b>0$ より $b$ も最大値を取る.よって, $b$ の最大値は $\sqrt{2}$ である.
\end{enumerate}

% ====================

\end{document}
