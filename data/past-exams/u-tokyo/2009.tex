\documentclass[a4paper]{ltjsarticle}

\usepackage[dvipdfmx]{graphicx}
\usepackage[dvipdfmx,hidelinks,pdfusetitle]{hyperref}
\hypersetup{
    colorlinks=false,
    bookmarksnumbered=true,
    pdfborder={0 0 0},
    bookmarkstype=toc
}
\usepackage[nobreak]{cite}
\usepackage{pxjahyper}
\usepackage{amsmath}
\usepackage{tikz}

\usetikzlibrary{datavisualization}
\usetikzlibrary{positioning}
\usetikzlibrary{shapes.geometric, shapes.misc}
\usetikzlibrary{patterns}
\usetikzlibrary{calc}

\begin{document}

% ====================

\begin{itembox}[l]{東京大学 2009年 文科 第1問}
    座標平面において原点を中心とする半径2の円を $C_1$ とし,点 $(1,\ 0)$ を中心とする半径1の円を $C_2$ とする.また,点 $(a,\ b)$ を中心とする半径 $t$ の円 $C_3$ が, $C_1$ に内接し,かつ $C_2$ に外接すると仮定する.ただし, $b$ は正の実数とする.

    \begin{enumerate}[label=(\arabic*)]
        \item $a,\ b$ を $t$ を用いて表せ.また,$t$ が取り得る値の範囲を求めよ.

        \item $t$ が(1)で求めた範囲を動くとき, $b$ の最大値を求めよ.
    \end{enumerate}
\end{itembox}

\begin{enumerate}[label=(\arabic*)]
    \item $C_2$ の中心を $O_2$,$C_3$ の中心を $O_3$ とし, $C_1$ と $C_3$ の接点を $A$,$C_3$ の接点を $B$ とする.

          $O$,$O_3$ および $A$ は一直線上にあるので,

          \begin{align}
              \text{OO}_3             & =2-t\label{eq:1}     \\
              \therefore\quad a^2+b^2 & =(2-t)^2\label{eq:2}
          \end{align}

          また,$O_2$,$B$ および $O_3$ は一直線上にあるので,

          \begin{align}
              \text{O}_2\text{O}_3        & =t+1\label{eq:3}     \\
              \therefore\quad (a-1)^2+b^2 & =(t+1)^2\label{eq:4}
          \end{align}

          \eqref{eq:2}$-$\eqref{eq:4} より,

          \begin{align*}
              2a-1              & =-6t+3 \\
              \therefore\quad a & =-3t+2
          \end{align*}

          上式を\eqref{eq:2}に代入すると,

          \begin{align}
              b^2=(2-t)^2-(-3t+2)^2 & =-8t^2+8t\label{eq:5}     \\
              \therefore\quad b     & =\sqrt{-8t^2+8t}\nonumber
          \end{align}

          \eqref{eq:1},\eqref{eq:3}および\eqref{eq:5}がすべて正になる条件より,$0<t<1$

    \item $f(t)=b^2$ を考える.

          \begin{equation*}
              f(t)=-8t^2+8t=-8\left(t-\frac{1}{2}\right)^2+2
          \end{equation*}

          $0<t<1$ より,$f(t)$ は $t=\dfrac{1}{2}$ で最大値2をとる.このとき,$b>0$ より $b$ も最大値を取る.よって, $b$ の最大値は $\sqrt{2}$ である.
\end{enumerate}

% ====================

\begin{itembox}[l]{東京大学 2009年 文科 第4問}
    2次以下の整式 $f(x)=ax^2+bx+c$ に対し

    \begin{equation*}
        S=\int_{0}^{2}|f'(x)|dx
    \end{equation*}

    を考える.

    \begin{enumerate}[label=(\arabic*)]
        \item $f(0)=0$,$f(2)=2$ のとき $S$ を $a$ の関数として表せ.

        \item $f(0)=0$,$f(2)=2$ を満たしながら $f$ が変化するとき,$S$ の最小値を求めよ.
    \end{enumerate}
\end{itembox}

$f'(x)=2ax+b$ より,

\begin{equation*}
    S=\int_{0}^{2}|2ax+b|dx
\end{equation*}

\begin{enumerate}[label=(\arabic*)]
    \item

          \begin{equation*}
              \left\{\begin{aligned}
                  f(0) & =c=0       \\
                  f(2) & =4a+2b+c=2
              \end{aligned}\right.
          \end{equation*}

          よって,

          \begin{equation*}
              b=1-2a,quad\therefore\quad S=\int_{0}^{2}|2ax+1-2a|dx
          \end{equation*}

          \begin{enumerate}[label=(\roman*)]
              \item $a\leq -\frac{1}{2}$ のとき,

                    \begin{figure}[!ht]
                        \centering
                        \includegraphics[width=0.5\textwidth]{2009-2.png}
                    \end{figure}

                    \begin{align*}
                        S & =\frac{1}{2}(1-2a)\left(1-\frac{1}{2a}\right)+\frac{1}{2}(-1-2a)\left(1+\frac{1}{2a}\right) \\
                          & =-2a-\frac{1}{2a}
                    \end{align*}

              \item $a\geq \frac{1}{2}$ のとき,

                    \begin{figure}[!ht]
                        \centering
                        \includegraphics[width=0.5\textwidth]{2009-3.png}
                    \end{figure}

                    \begin{align*}
                        S & =\frac{1}{2}(-1+2a)\left(1-\frac{1}{2a}\right)+\frac{1}{2}(1+2a)\left(1+\frac{1}{2a}\right) \\
                          & =2a+\frac{1}{2a}
                    \end{align*}

              \item $-\frac{1}{2}\leq a\leq\frac{1}{2}$ のとき,

                    \begin{figure}[!ht]
                        \centering
                        \includegraphics[width=0.5\textwidth]{2009-4.png}
                    \end{figure}

                    \begin{align*}
                        S & =\frac{\{(1-2a)+(1+2a)\}\cdot 2}{2} \\
                          & =2
                    \end{align*}
          \end{enumerate}

          以上より,

          \begin{equation*}
              S=\left\{\begin{aligned}
                   & -2a-\frac{1}{2a} &  & \left(a\leq -\frac{1}{2}\right)                 \\
                   & 2                &  & \left(-\frac{1}{2}\leq a\leq \frac{1}{2}\right) \\
                   & 2a+\frac{1}{2a}  &  & \left(a\geq \frac{1}{2}\right)
              \end{aligned}\right.
          \end{equation*}

    \item $S$ は $a$ の偶関数であるので,$a\geq 0$ の範囲で考える.$a\geq \frac{1}{2}$ のとき,相加平均・相乗平均の関係より,

          \begin{equation*}
              S=2a+\frac{1}{2a}\geq 2\sqrt{2a\cdot\frac{1}{2a}}=2
          \end{equation*}

          一方,$0\leq a\leq \frac{1}{2}$ のとき,$S=2$ であることから,$S$ の最小値は2である.
\end{enumerate}

% ====================

\begin{itembox}[l]{東京大学 2009年 理科 第4問}
    $a$ を正の実数とし,空間内の2つの円板

    \begin{align*}
        D_1 & =\{(x,\ y,\ z)\mid x^2+y^2\leq 1,\ z=a\}  \\
        D_2 & =\{(x,\ y,\ z)\mid x^2+y^2\leq 1,\ z=-a\}
    \end{align*}

    を考える.$D_1$ を $y$ 軸の周りに $180^\circ$ 回転して $D_2$ に重ねる.ただし回転は $z$ 軸の正の部分を $x$ 軸の正の方向に傾ける向きとする.この回転の間に $D_1$ が通る部分を $E$ とする.$E$ の体積を $V(a)$ とし,$E$ と $\{(x,\ y,\ z)\mid x\geq 0\}$ との共通部分の体積を $W(a)$ とする.

    \begin{enumerate}[label=(\arabic*)]
        \item $W(a)$ を求めよ.

        \item $\displaystyle\lim_{a\to\infty}V(a)$ を求めよ.
    \end{enumerate}
\end{itembox}

\begin{enumerate}[label=(\arabic*)]
    \item
          \begin{figure}[!ht]
              \centering
              \includegraphics[width=0.5\textwidth]{2009-5.png}
          \end{figure}

          $-1\leq t\leq 1$ に対し,平面 $y=t$ での $W(a)$ の断面積を $S(t)$ とおくと,

          \begin{equation*}
              S(t)=\frac{1}{2}\pi\left\{(\sqrt{1+a^2-t^2})^2-a^2\right\}=\frac{1}{2}\pi(1-t^2)
          \end{equation*}

          したがって,

          \begin{align*}
              W(a) & =\int_{-1}^{1}S(t)dt                  \\
                   & =2\int_{0}^{1}\frac{1}{2}\pi(1-t^2)dt \\
                   & =\pi\left[t-\frac{1}{3}t^3\right]     \\
                   & =\frac{2}{3}\pi
          \end{align*}

    \item $E$ と $\{(x,\ y,\ z)\mid x\leq 0\}$ との共通部分の体積を $W'(a)$ とすると,

          \begin{align*}
               & 0<W'(a)<2\cdot 2(\sqrt{1+a^2}-a)             \\
               & \lim_{a\to\infty}2\cdot 2(\sqrt{1+a^2}-a)    \\
               & =\lim_{a\to\infty}\frac{4}{\sqrt{1+a^2}+a}=0
          \end{align*}

          したがって,はさみうちの原理より,

          \begin{equation*}
              \lim_{a\to\infty}W'(a)=0
          \end{equation*}

          よって,

          \begin{equation*}
              \lim_{a\to\infty}V(a)=\lim_{a\to\infty}\{W(a)+W'(a)\}=\frac{2}{3}\pi
          \end{equation*}

          \begin{figure}[!ht]
              \centering
              \includegraphics[width=0.5\textwidth]{2009-6.png}
          \end{figure}

          [別解]

          \begin{figure}[!ht]
              \centering
              \includegraphics[width=0.5\textwidth]{2009-7.png}
          \end{figure}

          $V(a)-W(a)$ を考える.$V(a)-W(a)$ は上図の斜線部を $z$ 軸を軸にして一周させた回転体の体積と等しい.よって

          \begin{align*}
              V(a)-W(a) & =\pi\int_{a}^{b}(b^2-z^2)dz                    \\
                        & =\pi\left[b^2z-\frac{1}{3}z^3\right]_{a}^{b}   \\
                        & =\pi\left[b^2(b-a)-\frac{1}{3}(b^3-a^3)\right] \\
                        & =\frac{\pi}{3}(b-a)(2b^2-ab-a^2)
          \end{align*}

          ここで,

          \begin{align*}
              \lim_{a\to\infty}b-a         & =\lim_{a\to\infty}\sqrt{a^2+1}-a                        \\
                                           & =\lim_{a\to\infty}\frac{1}{\sqrt{a^2+1}+a}              \\
                                           & =0                                                      \\
              \lim_{a\to\infty}2b^2-ab-a^2 & =\lim_{a\to\infty}2(a^2+1)-a\sqrt{a^2+1}-a^2            \\
                                           & =\lim_{a\to\infty}a^2-a\sqrt{a^2+1}+2                   \\
                                           & =\lim_{a\to\infty}a\cdot\frac{-1}{a+\sqrt{a^2+1}}+2     \\
                                           & =\lim_{a\to\infty}-\frac{1}{1+\sqrt{1+\frac{1}{a^2}}}+2 \\
                                           & =\frac{3}{2}
          \end{align*}

          以上より,

          \begin{align*}
               & \lim_{a\to\infty}V(a)-W(a)=\frac{\pi}{3}\cdot 0\cdot\frac{3}{2}=0          \\
               & \therefore\quad \lim_{a\to\infty}V(a)=\lim_{a\to\infty}W(a)=\frac{2}{3}\pi
          \end{align*}
\end{enumerate}

\end{document}
