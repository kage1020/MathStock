\documentclass[a4paper]{ltjsarticle}

\usepackage[dvipdfmx]{graphicx}
\usepackage[dvipdfmx,hidelinks,pdfusetitle]{hyperref}
\hypersetup{
    colorlinks=false,
    bookmarksnumbered=true,
    pdfborder={0 0 0},
    bookmarkstype=toc
}
\usepackage[nobreak]{cite}
\usepackage{pxjahyper}
\usepackage{amsmath}
\usepackage{tikz}

\usetikzlibrary{datavisualization}
\usetikzlibrary{positioning}
\usetikzlibrary{shapes.geometric, shapes.misc}
\usetikzlibrary{patterns}
\usetikzlibrary{calc}

\begin{document}

% ====================

\begin{itembox}[l]{東京大学 2011年 文科 第4問}
    座標平面上の1点 P$\left(\frac{1}{2},\ \frac{1}{4}\right)$ をとる.放物線 $y=x^2$ 上の2点 Q$(\alpha,\ \alpha^2)$,R$(\beta,\ \beta^2)$ を,3点P,Q,RがQRを底辺とする二等辺三角形をなすように動かすとき,$\triangle$PQR の重心 G$(X,\ Y)$ の軌跡を求めよ.
\end{itembox}

\begin{figure}[!ht]
    \centering
    \includegraphics[width=0.5\textwidth]{2011-1.png}
\end{figure}

二等辺三角形の条件より,

\begin{align*}
     & \text{PQ}=\text{PR}(>0)                                                                                       \\
     & \Leftrightarrow \text{PQ}^2=\text{PR}^2                                                                       \\
     & \Leftrightarrow (\frac{1}{2}-\alpha)^2+(\frac{1}{4}-\alpha^2)^2=(\frac{1}{2}-\beta)^2+(\frac{1}{4}-\beta^2)^2 \\
     & \Leftrightarrow \alpha^4+\frac{1}{2}\alpha^2-\alpha=\beta^4+\frac{1}{2}\beta^2-\beta                          \\
     & \Leftrightarrow (\alpha^4-\beta^4)+\frac{1}{2}(\alpha^2-\beta^2)-(\alpha-\beta)=0                             \\
     & \Leftrightarrow (\alpha-\beta)\left\{(\alpha^2+\beta^2)(\alpha+\beta)+\frac{1}{2}(\alpha+\beta)-1\right\}=0
\end{align*}

となる.ここで,$\alpha\neq\beta$ のとき,三角形PQRができないから,$\alpha-\beta\neq 0$ となり,

\begin{equation}
    (\alpha^2+\beta^2)(\alpha+\beta)+\frac{1}{2}(\alpha+\beta)-1=0\label{eq:1}
\end{equation}

を得る.ここで,$\alpha+\beta=s$,$\alpha\beta=t$ とおくと,$\alpha$,$\beta$ はの2次方程式 $x^2-sx+t=0$ の異なる2実数解だから,

\begin{equation}
    s^2-4t>0\label{eq:2}
\end{equation}

となる.$s$,$t$ を用いて\eqref{eq:1}を書き換えると,

\begin{equation*}
    (s^2-2t)s+\frac{1}{2}s-1=0
\end{equation*}

$s=0$ とすると, $-1=0$ となり,不適である.したがって $s\neq 0$ となる.ゆえに

\begin{align}
     & s^2-2t=\frac{1}{s}\left(1-\frac{1}{2}s\right)\nonumber                \\
     & \therefore\quad t=\frac{1}{2}s^2-\frac{1}{2s}+\frac{1}{4}\label{eq:3}
\end{align}

である.ここで,

\begin{align}
    X & =\frac{1}{3}\left\{\frac{1}{2}+(\alpha+\beta)\right\}=\frac{1}{3}\left(\frac{1}{2}+s\right)\label{eq:4} \\
    Y & =\frac{1}{3}\left\{\frac{1}{4}+(\alpha^2+\beta^2)\right\}                                               \\
      & =\frac{1}{3}\left\{\frac{1}{4}+(\alpha+\beta)^2-2\alpha\beta\right\}\nonumber                           \\
      & =\frac{1}{3}\left\{\frac{1}{4}+s^2-2t\right\}\nonumber                                                  \\
      & =\frac{1}{3}\left(\frac{1}{s}-\frac{1}{4}\right)\label{eq:5}
\end{align}

\eqref{eq:4}より,

\begin{equation}
    s=3X-\frac{1}{2}\label{eq:6}
\end{equation}

これを\eqref{eq:5}に代入して,

\begin{equation}
    Y=\frac{1}{3}\left(\frac{1}{3X-\frac{1}{2}}-\frac{1}{4}\right)=\frac{1}{9}\cdot\frac{1}{X-\frac{1}{6}}-\frac{1}{12}\label{eq:7}
\end{equation}

ここで,\eqref{eq:2}に\eqref{eq:3}を代入して,

\begin{align}
     & -s^2+\frac{2}{s}-1>0\nonumber                     \\
     & \Longleftrightarrow\quad s^4+s^2-2s<0\nonumber    \\
     & \Longleftrightarrow\quad s(s-1)(s^2+s+2)<0\tag{*}
\end{align}

ここで,$s^2+s+2=\left(s+\frac{1}{2}\right)^2+\frac{7}{4}>0$ より,

\begin{equation*}
    (*)\quad\Longleftrightarrow\quad 0<s<1
\end{equation*}

これに\eqref{eq:6}を代入して,

\begin{align}
     & 0<3X-\frac{1}{2}<1\nonumber                          \\
     & \therefore\quad\frac{1}{6}<X<\frac{1}{2}\label{eq:8}
\end{align}

ゆえに,求める軌跡は\eqref{eq:7},\eqref{eq:8}より,曲線 $y=\frac{1}{9}\cdot\frac{1}{x-\frac{1}{6}}-\frac{1}{12}$ の $\frac{1}{6}<x<\frac{1}{2}$ の部分

[別解]

二等辺三角形の条件より,

\begin{align*}
     & \text{PQ}=\text{PR}(>0)                                                                                       \\
     & \Leftrightarrow \text{PQ}^2=\text{PR}^2                                                                       \\
     & \Leftrightarrow (\frac{1}{2}-\alpha)^2+(\frac{1}{4}-\alpha^2)^2=(\frac{1}{2}-\beta)^2+(\frac{1}{4}-\beta^2)^2 \\
     & \Leftrightarrow \alpha^4+\frac{1}{2}\alpha^2-\alpha=\beta^4+\frac{1}{2}\beta^2-\beta                          \\
     & \Leftrightarrow (\alpha^4-\beta^4)+\frac{1}{2}(\alpha^2-\beta^2)-(\alpha-\beta)=0                             \\
     & \Leftrightarrow (\alpha-\beta)\left\{(\alpha^2+\beta^2)(\alpha+\beta)+\frac{1}{2}(\alpha+\beta)-1\right\}=0
\end{align*}

ここで,$\alpha=\beta$ のとき,三角形PQRができないので,$\alpha-\beta\neq 0$ .よって,

\begin{align}
     & (\alpha^2+\beta^2)(\alpha+\beta)+\frac{1}{2}(\alpha+\beta)-1=0\nonumber         \\
     & \Leftrightarrow \alpha^2+\beta^2=\frac{1}{\alpha+\beta}-\frac{1}{2}\label{eq:9}
\end{align}

G$(X,\ Y)$ は $\triangle$PQR の重心なので,

\begin{align}
    X & =\frac{\frac{1}{2}+(\alpha+\beta)}{3}=\frac{1}{3}\left(\alpha+\beta+\frac{1}{2}\right)\label{eq:10}             \\
    Y & =\frac{\frac{1}{4}+\alpha^2+\beta^2}{3}=\frac{1}{3}\left(\frac{1}{\alpha+\beta}-\frac{1}{4}\right)\label{eq:11}
\end{align}

これより,

\begin{equation}
    \left\{\begin{aligned}
         & \alpha+\beta=\frac{1}{2}(6X-1)        \\
         & \frac{1}{\alpha+\beta}=3Y+\frac{1}{4}
    \end{aligned}\right.\label{eq:12}
\end{equation}

となる.$\alpha-\beta\neq 0$ から $6X-1\neq 0$ であり,

\begin{equation}
    3Y+\frac{1}{4}=\frac{2}{6X-1}\quad\Longleftrightarrow\quad Y=\frac{1}{9(X-\frac{1}{6})}-\frac{1}{12}\label{eq:13}
\end{equation}

を得る.ここで,\eqref{eq:10},\eqref{eq:11}から,

\begin{equation*}
    \left\{\begin{aligned}
        \alpha+\beta     & =\frac{1}{2}(6X-1) \\
        \alpha^2+\beta^2 & =3Y-\frac{1}{4}
    \end{aligned}\right.
\end{equation*}

であり,

\begin{equation*}
    \alpha\beta = \frac{1}{2}\left\{\left(\alpha+\beta\right)^2-\left(\alpha^2+\beta^2\right)\right\}
\end{equation*}

であるから,\eqref{eq:12}は,

\begin{equation}
    \left\{\begin{aligned}
         & \alpha+\beta=\frac{6X-1}{2}         \\
         & \alpha\beta=\frac{18X^2-6X-6Y+1}{4}
    \end{aligned}\right.\label{eq:14}
\end{equation}

と同値である.\eqref{eq:14}を満たす異なる実数 $\alpha$,$\beta$ が存在するための $(X,\ Y)$ の条件を求める.これは $t$ の2次方程式

\begin{equation}
    t^2-\frac{6X-1}{2}t+\frac{18X^2-6X-6Y+1}{4}=0\label{eq:15}
\end{equation}

が異なる2つの実数解を持つための条件であり,\eqref{eq:15}についての判別式 $D>0$ なので,

\begin{align}
     & \left(\frac{6X-1}{2}\right)^2-4\cdot\frac{18X^2-6X-6Y+1}{4}>0\nonumber                       \\
     & \Longleftrightarrow\quad Y>\frac{3}{2}X^2-\frac{1}{2}X+\frac{1}{8}\nonumber                  \\
     & \Longleftrightarrow\quad Y>\frac{3}{2}\left(X-\frac{1}{6}\right)^2+\frac{1}{12}\label{eq:16}
\end{align}

よって,$(X,\ Y)$ は\eqref{eq:13}かつ\eqref{eq:16}を満たす.逆に\eqref{eq:13}かつ\eqref{eq:16}を満たす任意の $X, Y$ に対して $t$ の2次方程式\eqref{eq:15}の解を $\alpha$,$\beta$ とすると,\eqref{eq:14}が成り立ち,\eqref{eq:13},\eqref{eq:14}から\eqref{eq:9},\eqref{eq:10},\eqref{eq:11}を得る.よって,放物線上に点 Q$(\alpha,\alpha^2)$,点 R$(\beta,\beta^2)$ をとると,確かに PQ$=$PR が成り立ち,かつ $(X,\ Y)$ は $\triangle$PQR の重心になっている.ゆえに,\eqref{eq:13}かつ\eqref{eq:16}を満たす $(X,\ Y)$ が重心 G の軌跡である.

曲線 $y=\frac{1}{9}\cdot\frac{1}{x-\frac{1}{6}}-\frac{1}{12}$ の $\frac{1}{6}<x<\frac{1}{2}$ の部分

% ====================

\end{document}
