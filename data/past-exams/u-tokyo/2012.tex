\documentclass[a4paper]{ltjsarticle}

\usepackage[dvipdfmx]{graphicx}
\usepackage[dvipdfmx,hidelinks,pdfusetitle]{hyperref}
\hypersetup{
    colorlinks=false,
    bookmarksnumbered=true,
    pdfborder={0 0 0},
    bookmarkstype=toc
}
\usepackage[nobreak]{cite}
\usepackage{pxjahyper}
\usepackage{amsmath}
\usepackage{tikz}

\usetikzlibrary{datavisualization}
\usetikzlibrary{positioning}
\usetikzlibrary{shapes.geometric, shapes.misc}
\usetikzlibrary{patterns}
\usetikzlibrary{calc}

\begin{document}

% ====================

\begin{itembox}[l]{東京大学 2011年 文科 第4問}
    座標平面上の放物線 $C$ を $y=x^2+1$ で定める.$s, t$ は実数とし $t<0$ を満たすとする.点 $(s, t)$ から放物線 $C$ へ引いた接線を $l_1, l_2$ とする.

    \begin{enumerate}[label=(\arabic*)]
        \item $l_1, l_2$ の方程式を求めよ.
        \item $a$ を正の実数とする.放物線 $C$ と直線 $l_1, l_2$ で囲まれる領域の面積が $a$ となる $(s, t)$ を全て求めよ.
    \end{enumerate}
\end{itembox}

\begin{enumerate}[label=(\arabic*)]
    \item
          $y=x^2+1$ 上の点 P$(p,\ p^2+1)$ における接線の方程式は,

          \begin{align}
               & y=2p(x-p)+p^2+1\nonumber                \\
               & \therefore\quad y=2px-p^2+1\label{eq:1}
          \end{align}

          \eqref{eq:1}が点 $(s, t)$ を通るとき,

          \begin{align}
               & t=2ps-p^2+1\nonumber                      \\
               & \therefore\quad p^2-2ps+t-1=0\label{eq:2}
          \end{align}

          \eqref{eq:2}より,$(s,\ t)$ から $C$ に引いた接線の接点の $x$ 座標 $p$ は

          \begin{equation}
              p=s\pm\sqrt{s^2-t+1}\label{eq:3}
          \end{equation}

          したがって,$l_1, l_2$ の方程式は\eqref{eq:1}に\eqref{eq:3}を代入して,

          \begin{align}
              l_1\colon & y=2(s-\sqrt{s^2-t+1})x-(s-\sqrt{s^2-t+1})^2+1\label{eq:4} \\
              l_2\colon & y=2(s+\sqrt{s^2-t+1})x-(s+\sqrt{s^2-t+1})^2+1\label{eq:5}
          \end{align}

    \item
          \begin{equation*}
              \alpha=s-\sqrt{s^2-t+1},\ \beta=s+\sqrt{s^2-t+1}
          \end{equation*}

          とおくと,\eqref{eq:4},\eqref{eq:5}はそれぞれ

          \begin{align}
              y & =2\alpha x-\alpha^2+1\label{eq:6} \\
              y & =2\beta x-\beta^2+1\label{eq:7}
          \end{align}

          となり,\eqref{eq:6},\eqref{eq:7}の交点の $x$ 座標は,$x=\frac{\alpha+\beta}{2}$ である.したがって,$C$ と $l_1, l_2$ で囲まれる部分の面積を $S$ とすると,

          \begin{figure}[!ht]
              \centering
              \includegraphics[width=0.5\textwidth]{2012-1.png}
          \end{figure}

          \begin{align*}
              S & =\int_\alpha^{\frac{\alpha+\beta}{2}}\left\{x^2+1-(2\alpha x-\alpha^2+1)\right\}dx+\int_{\frac{\alpha+\beta}{2}}^\beta\left\{x^2+1-(2\beta x-\beta^2+1)\right\}dx \\
                & =\int_\alpha^{\frac{\alpha+\beta}{2}}(x-\alpha)^2dx+\int_{\frac{\alpha+\beta}{2}}^\beta(x-\beta)^2dx                                                              \\
                & =\left[\frac{1}{3}(x-\alpha)^3\right]_\alpha^{\frac{\alpha+\beta}{2}}+\left[\frac{1}{3}(x-\beta)^3\right]_{\frac{\alpha+\beta}{2}}^\beta                          \\
                & =\frac{1}{12}(\beta-\alpha)^3                                                                                                                                     \\
                & =\frac{1}{12}(2\sqrt{s^2-t+1})^3                                                                                                                                  \\
                & =\frac{2}{3}(s^2-t+1)^{\frac{3}{2}}
          \end{align*}

          $S=a$ のとき,

          \begin{align*}
               & \frac{2}{3}(s^2-t+1)^{\frac{3}{2}}=a                            \\
               & \therefore\quad s^2-t+1=\left(\frac{3}{2}a\right)^{\frac{2}{3}} \\
               & \therefore\quad t=s^2+1-\left(\frac{3}{2}a\right)^{\frac{2}{3}}
          \end{align*}

          \begin{enumerate}[label=(\roman*)]
              \item $1-\left(\frac{3}{2}a\right)^{\frac{2}{3}}\geq 0\Longleftrightarrow 0<a\leq\frac{2}{3}$ のとき,

                    \begin{equation*}
                        t=s^2+1-\left(\frac{3}{2}a\right)^{\frac{2}{3}}\geq 0
                    \end{equation*}

                    より,$t<0$ を満たす $(s,\ t)$ は存在しない.
              \item $1-\left(\frac{3}{2}a\right)^{\frac{2}{3}}<0\Longleftrightarrow a>\frac{2}{3}$ のとき,

                    $t=s^2+1-\left(\frac{3}{2}a\right)^{\frac{2}{3}}$ かつ $t<0$ を満たす $(s,\ t)$ が求める部分である.
          \end{enumerate}
\end{enumerate}

[別解]

\begin{enumerate}[label=(\arabic*)]
    \item 点 $(s,\ t)$ を通る傾き $a$ の直線は,

          \begin{equation*}
              y=a(x-s)+t=ax-as+t
          \end{equation*}

          と表せる.これが,放物線 $C$ と接するとき,

          \begin{equation*}
              x^2+1=ax^as+t\quad\Longleftrightarrow\quad x^2-ax+1+as-t=0
          \end{equation*}

          より,上式の判別式を $D$ とすると,$D=0$ が成り立てばよい.したがって,

          \begin{align*}
              D                                       & =0                                             \\
              \Longleftrightarrow\quad a^2-4(1+as-t)  & =0                                             \\
              \Longleftrightarrow\quad a^2-4sa+4(t-1) & =0                                             \\
              \therefore\quad a                       & =2s\pm\sqrt{4s^2-4(t-1)}=2(s\pm\sqrt{s^2-t+1})
          \end{align*}

          と求まる.したがって,$l_1, l_2$ の方程式は,

          \begin{equation*}
              y=2(s\pm\sqrt{s^2-t+1})x-2(s\pm\sqrt{s^2-t+1})s+t
          \end{equation*}

          となる.
\end{enumerate}

% ====================

\end{document}
